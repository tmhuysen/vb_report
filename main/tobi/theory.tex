\section{Framework}
Conventions:
\begin{itemize}
\item Greek letters $\tau$ and $\sigma$ indicate spin functions.
\item $\phi$ and lowercase subscript letters ($p,q,r,s$) indicate the non-orthogonal atomic orbitals.  
\end{itemize}    

Let us define $\hat{a}^\dagger_p$ and $\hat{a}_q$ as elementary creation and annihilation operators of an non-orthogonal spatial basis. 
Evaluating the anti-commutator on the vacuum state:
    \begin{align}
        \ev{\comm*{\hat{a}^\dagger_p}{\hat{a}_q}_+}{\text{vac}} &= \ev{\hat{a}_q \hat{a}^\dagger_p}{\text{vac}} \\
        &= \braket{\phi_q}{\phi_p} \\
        &= {S}_{q p} \\
         \qquad \comm{\hat{a}^\dagger_p}{\hat{a}_q}_+ & = S_{q p} \label{eq:sq_overlap}
\end{align}
With spin in the picture the non-relativistic framework allows us to define elementary operators in the non-orthogonal spin orbitals basis as:
    \begin{align}
        \comm{\hat{a}_{p\sigma}^\dagger}{\hat{a}_{q \tau}}_+ &= S_{q p} \delta_{\sigma \tau} \\
        \comm{\hat{a}_{p\sigma}^\dagger}{\hat{a}^\dagger_{q \tau}}_+ &= 0 \\
        \comm{\hat{a}_{p\sigma}}{\hat{a}_{q \tau}}_+ &= 0 
    \end{align} 

A valence bond structure can then be expressed as a summation of creation operators on a vacuum state (with $k_{p\sigma}$ 0 if not occupied in the given state and 1 if occupied):
    \begin{equation} \label{eq:ON_creators_vacuum}
        \ket{\vb{k}} = \prod_\sigma \prod_p^K (\hat{a}^\dagger_{p\sigma})^{k_{p\sigma}} \ket{\text{vac}} 
    \end{equation}
    \begin{equation} \label{eq:ON_creators_vacuum}
        \psi_{\text{vb}} = \sum_k \ket{\vb{k}} 
    \end{equation}
For given valence wave function  we can then write:
    \begin{equation} \label{eq:ON_creators_vacuum}
        \Psi_{\text{vb}} = \sum_i c_i \psi_{i}
    \end{equation}


The electronic Hamiltonian in restricted space can then be written as:         
  \begin{equation}\label{eq:ham_sf}
          \hat{\mathcal{H}}_\text{elec} =  \sum_{pq}^K h_{pq} \hat{E}_{pq} + \frac{1}{2}  \sum_{pqrs}^K g_{pqrs} \hat{e}_{pqrs}
 \end{equation}        
 
  Where $\hat{E}_{pq}$ is the singlet one-electron excitation operator:
        \begin{align}
            \hat{E}_{pq} &= \hat{a}^\dagger_{p\alpha} \hat{a}_{q\alpha} + \hat{a}^\dagger_{p \beta} \hat{a}_{q \beta} \\
            &= \hat{E}^\alpha_{pq} + \hat{E}^\beta_{pq} \label{eq:E_pq_alpha_beta} \\
            &= \sum_\sigma \hat{a}^\dagger_{p \sigma} \hat{a}_{q \sigma} 
\end{align} 

        And $\hat{e}_{pqrs}$ is a singlet two-electron excitation operator:
        \begin{align}
            \hat{e}_{pqrs} &= \hat{E}_{pq} \hat{E}_{rs} - \delta_{qr} \hat{E}_{ps} \\
            &= \sum_{\sigma \tau} \hat{a}^\dagger_{p \sigma} \hat{a}^\dagger_{r \tau} \hat{a}_{s \tau} \hat{a}_{q \sigma} \label{eq:two_electron_excitation_operator}
        \end{align}
        
